\documentclass[aspectratio=169]{beamer}
\usetheme{metropolis}
\title{InterIA --- Invariants \& Garde-fous (Robin Ready)}
\author{Alexandre Couret \& InterIA}
\date{\today}
\begin{document}
\begin{frame}{Pourquoi InterIA ?}
\small
Principe : troncatures $\to$ invariant (lambda/pi) $\to$ garde-fous falsifiables (NonTautology / Progression / RMT).\\
Cadre : Mod 30 = temps ; Fermat = espace ; Riemann = union (ligne critique/spectres).\\
Programme RH : A (Gram PSD) ; B (T2 pond\'er\'ee) ; C (RMT/TDA) ; D (Jensen) ; E (op\'erateur).\\
Plomberie : schemas JSON, Make, CI, dashboards, snapshots.
\end{frame}
\end{document}
