\documentclass[11pt]{article}
\usepackage[T1]{fontenc}
\usepackage[utf8]{inputenc}
\usepackage[a4paper,margin=2.5cm]{geometry}
\usepackage{amsmath,amssymb,amsthm,mathtools}
\usepackage{hyperref}
\hypersetup{colorlinks=true,linkcolor=black,urlcolor=blue}
\title{Une Preuve de la Nature Non-Pr\'etentieuse des Ensembles de D\'efauts $D_k$ par la M\'ethode de la Dispersion}
\author{Alexandre Couret \& InterIA}
\date{\today}
\numberwithin{equation}{section}
\newtheorem{theorem}{Th\'eor\`eme}[section]
\newtheorem{proposition}[theorem]{Proposition}
\newtheorem{lemma}[theorem]{Lemme}
\newtheorem{definition}[theorem]{D\'efinition}
\newtheorem{remark}[theorem]{Remarque}
\begin{document}\maketitle
\section{Introduction et \'Enonc\'e de la Proposition}
\begin{proposition}\label{prop:dispersion_appendix}
La suite $(D_k)_{k\ge 1}$ g\'en\'er\'ee par $T_3$ est uniform\'ement non--pr\'etentieuse.
\end{proposition}
\section{Cadre pr\'etentieux}\begin{definition}
$\mathbb{D}(f,g;x)^2 := \sum_{p\le x}\frac{1-\mathrm{Re}(f(p)\overline{g(p)})}{p}$.
\end{definition}
\section{M\'ethode de la Dispersion}
\section{Structure Arithm\'etique de $D_k$}
\section{D\'emonstration de la Proposition \ref{prop:dispersion_appendix}}
\end{document}
